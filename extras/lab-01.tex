\documentclass{tufte-handout}

\usepackage{xcolor}

% set image attributes:
\usepackage{graphicx}
\graphicspath{ {images/} }

% set hyperlink attributes
\hypersetup{colorlinks}

% set table attributes
\usepackage{tabu}
\usepackage{booktabs}

% create environment for bottom paragraph:
\newenvironment{bottompar}{\par\vspace*{\fill}}{\clearpage}

% =======================================================

% define the title
\title{SOC 4015/5050: Lab-01 - Initial Data Cleaning }
\author{Christopher Prener, Ph.D.}
\date{Fall 2018}
% ======================================================
\begin{document}
% =======================================================
\maketitle % generates the title
% =======================================================

\section{Directions}
Complete all of the following questions using the data from the \texttt{testDriveR} package.\sidenote{See course website for latest details on installing \texttt{testDriveR}, which can be found on the Course Software page!} Your well-formatted R Notebook source (the \texttt{.Rmd} file) and \texttt{html} output should be \textit{ready} to be uploaded to your assignments repository by 4:15pm on Monday, September 10\textsuperscript{th}, 2017. We'll go through the submission itself together as a class.

\vspace{5mm}
\section{Analysis Development: Create a Project Folder System}
\begin{enumerate}
\item Using RStudio, add an R Project to a new directory named \texttt{Lab-01}. To do this, you will want to go to: \textsf{File $\triangleright$} {\color{red}\textsf{New Project}} \textsf{$\triangleright$ New Directory $\triangleright$ New Project} and save your new folder and R Project to your Desktop or another similar location where you can easily find it.
\item RStudio should automatically open your new project. Verify this by looking up at the righthand corner of RStudio's window - you should see a blue box icon with a dark blue \texttt{R} in it. Next to that should be the text \texttt{Lab-01}.
\item R Projects set something called the working directory, which is a critically important piece of programming that we'll continue to talk about this semester.
\item In the \textsf{Files} tab on the lower righthand side of RStudio's screen, add a New Folder using the \textsf{New Folder} button right below \textsf{Files}. Name this new folder \texttt{docs}.
\item Create a new notebook by going to \textsf{File $\triangleright$ New File $\triangleright$} {\color{red}\textsf{R Notebook}}. Save it within that \texttt{docs/} subdirectory you just created.
\item Edit the heading of your notebook so that it looks like so: \\
\begin{verbatim}
---
title: "Lab-01 Notebook"
author: "your name"
date: '(`r format(Sys.time(), "%B %d, %Y")`)'
output: 
  github_document: default
  html_notebook: default 
---
\end{verbatim}
Getting the backticks correct in the \texttt{date:} field is tricky - look at the replication to see how it is done!
\item Use RMarkdown syntax to create your first assignment notebook! Make sure it has an introductory section, a section for loading packages, a section for loading data, and a section for each part below. These sections should be second-level headings (e.g. \texttt{\#\# Introduction}). In Both Part 1 and Part 2, use third level headings to designate question numbers (e.g. \texttt{\#\#\# Question 9}).
\item When you are done, ``knit'' your document by clicking the \textsf{Knit} button in the toolbar at the top of the notebook.
\end{enumerate}

\vspace{5mm}
\section{Part 1: Cleaning Data}
\textit{Use the \texttt{auto17} data frame saved in the \texttt{testDriveR} package and make the following changes using ``piped'' code:}
\begin{enumerate}
\setcounter{enumi}{8}
\item Extract observations for German cars (those manufactured by Audi, BMW, Mercedes-Benz, Porsche, and Volkswagon). 
\item Keep only the following variables: \texttt{id, mfrDivision, carLine, combFE, guzzlerStr, displ}
\item Rename the \texttt{mfrDivision} and \texttt{combFE} variables.\sidenote{Not sure what these variables measure? Type \texttt{?auto17} into the console of RStudio and scroll through the help file for the data set.}
\item Create a new logical variable that is \texttt{TRUE} if the vehicle is a guzzler (\texttt{guzzlerStr == "G"}) and is \texttt{FALSE} otherwise.
\item Re-order the data frame based on your re-named \texttt{combFE} variable from high to low.
\item Print the ``head'' of the data frame - what is the most fuel efficient German car for sale in the United States for model year 2017?
\item How many German cars in total are for sale in the United States for model year 2017?
\item How many German cars are ``gas guzzlers''?
\end{enumerate}

\vspace{5mm}
\section{Part 2: Plotting Data}
\textit{Use the your cleaned German car data to produce the following plots:}.
\begin{enumerate}
\setcounter{enumi}{16}
\item Create a bar plot of the logical ``gas guzzler'' variable you created.
\item Create a histogram of the average fuel efficiency variable.
\item Create a scatter plot of the average fuel efficiency and \texttt{displ} variables that (a) highlights ``gas guzzler'' vehicles and (b) uses the ``jitter'' positions adjustment.
\end{enumerate}

\vspace{5mm}
\section{Reminders}
Remember that a replication file will be posted on GitHub and linked to from the course website. I will also provide some screen shots of the analysis development section to help you navigate around RStudio's user interface. Don't forget to ``knit'' your notebook when you are done!

% ============================================================
\end{document}